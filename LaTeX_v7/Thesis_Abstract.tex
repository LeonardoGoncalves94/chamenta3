%%%%%%%%%%%%%%%%%%%%%%%%%%%%%%%%%%%%%%%%%%%%%%%%%%%%%%%%%%%%%%%%%%%%%%%%
%                                                                      %
%     File: Thesis_Abstract.tex                                        %
%     Tex Master: Thesis.tex                                           %
%                                                                      %
%     Author: Andre C. Marta                                           %
%     Last modified :  2 Jul 2015                                      %
%                                                                      %
%%%%%%%%%%%%%%%%%%%%%%%%%%%%%%%%%%%%%%%%%%%%%%%%%%%%%%%%%%%%%%%%%%%%%%%%

\section*{Abstract}

% Add entry in the table of contents as section
\addcontentsline{toc}{section}{Abstract}

\textit{Vehicle-to-everything} (V2X) communication has increasingly become the target of research and standardization efforts in Europe, America and Asia. This term refers to the exchanging of information between a vehicle and any entity that may affect the vehicle; such entities can be, for example, other vehicles, pedestrians or roadside units  i.e. semaphores, road barriers, signs, etc. V2X communication is an essential feature of autonomous vehicles in the future. Such vehicles are envisioned to increase road safety, driver comfort, and fuel economization through traffic efficiency. This work aims to survey the state-of-the-art of the V2X communication from a cyber-security point of view. We analyze different existing solutions regarding the \textit{Public Key Infrastructure} (PKI) and the standardization efforts needed to support a V2X driving environment.

\vfill

\textbf{\Large Keywords:} privacy, communication, vehicles, vehicular had-hoc network,  public key infrastructure,  digital signatures, digital certificates.

