%%%%%%%%%%%%%%%%%%%%%%%%%%%%%%%%%%%%%%%%%%%%%%%%%%%%%%%%%%%%%%%%%%%%%%%%
%                                                                      %
%     File: Thesis_Resumo.tex                                          %
%     Tex Master: Thesis.tex                                           %
%                                                                      %
%     Author: Andre C. Marta                                           %
%     Last modified :  2 Jul 2015                                      %
%                                                                      %
%%%%%%%%%%%%%%%%%%%%%%%%%%%%%%%%%%%%%%%%%%%%%%%%%%%%%%%%%%%%%%%%%%%%%%%%

\section*{Resumo}

% Add entry in the table of contents as section
\addcontentsline{toc}{section}{Resumo}
A \textit{comunicação entre veículos e coisas} (V2X) tem vindo a tornar-se um tópico de investigação e estandardização na Europa, América e Ásia. Este termo refere-se à troca de informação entre veículos e qualquer outra entidade que os possa afetar. For exemplo, outros veículos, pedestres, semáforos, sinalização de estrada, etc. A comunicação entre veículos e coisas é considerada uma característica essencial para atingir a automatização dos veículos no futuro. Estes veículos estão previstos para melhorar a segurança rodoviária, o conforto dos passageiros e autonomia de combustíveis a partir do aumento da eficiência do trânsito. Este trabalho tem como objetivo o estudo do estado da arte das comunicações V2X de um ponto de vista da ciber-segurança, assim como o desenvolvimento de um ambiente V2X. Toda a pesquisa e desenvolvimento apresentado nesa tese foi alcançado na empresa Multicert, onde foi proposto o tema, mostrando assim a sua relevância em contexto empresarial.
Vamos começar por analisar soluções existentes ao nível da \textit{Infraestrutura de Chaves Públicas} (ICP) e os esforços de estandardização necessários para suportar um ambiente V2X.
Por fim, apresentamos o nosso sistema que funciona num contexto web e tem por base os standards e a ICP
estudados.

\vfill

\textbf{\Large Palavras-chave:} privacidade, communicação, veículos, infraestrutura de chaves públicas, assinaturas digitais, certificados digitais.

